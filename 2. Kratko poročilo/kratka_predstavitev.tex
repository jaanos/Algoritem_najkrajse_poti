\documentclass[a4paper, 12 pt]{article}
\usepackage[utf8]{inputenc}
\usepackage[T1]{fontenc}
\usepackage[slovene]{babel}
\usepackage{lmodern}
\usepackage{amsmath}
\usepackage{amsfonts}
\usepackage{amssymb}
\usepackage{units}
\usepackage{eurosym}
\usepackage{pdfpages}
\usepackage{comment}
\usepackage{enumerate}
\usepackage{mathtools}
\usepackage{amsthm}

\theoremstyle{plain}
\newtheorem{izrek}{Izrek}
\theoremstyle{definition}
\newtheorem{definicija}{Definicija}
\theoremstyle{remark}
\newtheorem{opomba}{Opomba}

\begin{document}
\begin{titlepage}
		\begin{center}
		
		\large
		Univerza v Ljubljani\\
		\normalsize
		Fakulteta za matematiko in fiziko\\
		
		\small
		Finančna matematika - 1. stopnja\\
		
		\vspace{5 cm} 
		
		\large
		Sabrina Calcina in Jan Črne \\
		
		\vspace{0.5cm}
		\Large
		\textbf{Algoritmi in množice neodvisnosti za podatkovno vodene robustne probleme najkrajših poti  (kratko poročilo)}
		
		\vspace{0.5 cm}
		\normalsize
		Finančni praktikum
		
		\vspace{1.5cm}
		\normalsize
		Mentorja: prof. dr. Sergio Cabello in asist. dr. Janoš Vidali
		
		\vfill
		
		\large Ljubljana, november 2020
		
		\end{center}
\end{titlepage}

\section{Povzetek}

Ukvarjali se bomo, z  robustnimi problemi najkrajših poti. To so problemi, katerih cilj je najti pot, ki optimizira najslabše delovanje v neki množici negotovosti. Množica negotovosti je množica, ki vsebuje vse scenarije cen povezav, zgeneriranih ali zabeleženih na podlagi opažanj.
Predpostavka tega problema je, da so množice negotovosti podane s strani strokovnjakov, ki povedo obliko in velikost le te.
Množice negotovosti katere bova pregledala bodo enake tistim iz članka z naslovom Algorithms and uncertainty sets for data-driven robust shortest paths problem iz leta 2017. Meriteve v uporabljene v članku temeljijo na resničnih podatkih iz meritev prometa, v najinem delu bova podatke najverjetneje zgenerirala sama, ob nekaj vnaprej dolečinh predpostavkah. Na podlagi teh podatkov, ki torej vsebujejo cene povezav nekega usmerjenega grafa nato, po premisleku izberemo primerno množico negotovosti, tako da primerjamo uspešnost dobljenih robustnih poti znotraj in zunaj vzorca, znotraj vzorca so v članku uporabili 75 \% vseh nabranih podatkov. 
Na podlagi eksperimentov se nato osredotočimo na elipsoidne množice negotovosti in razvijemo nov algoritem s katerim nato iščemo najcenejšo povezavo med začetnim in končnim vozliščem.

\section{Uvod}
Za klasične probleme najkrajših poti v uličnih omrežjih so bile dosežene znatne pospešitve v primerjavi s standardnim Dijkstrovim algoritmom. Zahvala gre tehnikam novejših algoritmov, ki omogočajo uporabo informacij v realnem času, tudi v velikih omrežjih.
Kljub temu je večina robustnih problemov z najkrajšimi oz. najcenejšimi potmi časovno zahtevna in optimizacija v realnem času ni na voljo. Za oblikovanje robustnega problema je tako treba imeti opis vseh možnih in ustreznih scenarijev, na katere naj bi se pripravili.\newline

Literatura običajno predvideva, da je množica podana z neko mešanico predhodne obdelave podatkov in strokovnega znanja, ki ni del študije. To pomeni, da so bile preučevane različne vrste množic, vendar brez odgovora na to katera je prava izbira.\newline

Prvi članek, ki sledi drugačni perspektivi problema najkrajše poti, je izšel leta 2017. Gre za robustno optimizacijo, ki temelji na podatkih, kjer je gradnja negotove množice na podlagi surovih opazovanj del robustnega problema optimizacije. Na podlagi realnih opazovanj mesta Chicago izdelamo vrsto množic, izračunamo ustrezne robustne rešitve in izvedemo poglobljeno analizo njihove uspešnosti. To nam omogoča, da ugotovimo, katere množice so primerne za aplikacijo in katere ne.\newline

Kasneje se osredotočimo na primer elipsoidne negotovosti in zagotavljanje hitrejšega algoritma. 

\section{Opis problema in načrt za spopadanje z njim:}
\begin{itemize}
\item{Imamo nek usmerjen graf G = (V, A), v je množica vozljev, A množica povezav, za vsako povezavo poznamo njeno ceno, ki bo v naših primerih čas, potreben za prehod te povezave,}
\item{cilj je najti pot v grafu, ki zminimizira čas potreben za pot od začetnega do končnega oglišča, vendar za razliko od nam do zdaj znanih primerov, kjer so cene povezav podane eksaktno, so v našem primeru le te podane z množico opažanj teh povezav, to bodo najini surovi podatki}
\item{na glede na surove podatke lahko sedaj definiramo različne množice negotovosti (množica konvekse ovojnice, intervalska množica, elipsoidne množice negotovosti, ...)}
\item{na teh množicah lahko rešujemo drugačen problem, sedaj želimo poiskati takšno pot, da bomo zminimizirali časovno najugodnejšo pot v primeru da imamo najslabši možen scenarji. npr. želimo poiskati najkrajšo pot v mestu od ene točke do druge v primeru, kadar so ceste maksimalno zasedene, torej ko za vožnjo po njih porebujemo največ časa}
\item{za najim primer se bova osredotočila na izvajanje algoritma za iskanje teh poti na elipsoidinih množicah negotovosti, saj le te, kot je v članku omenjeno ponudijo najbolše razmerje med maksimalno in povprečno potjo, ter ponudijo zadovoljivo časovno zahtevnost algoritma.}
\end{itemize}








































\end{document}